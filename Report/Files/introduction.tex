Nowadays, more and more public but also private organizations wish to publish their data in order to allow other parties to perform research, such that all parties can benefit from the synergies and the resulting insights or in order to outsource a task related to the data to an external service provider. \\
However, in a time when personal data that is handled by both private and public actors is getting larger and more sophisticated, the threat against individual privacy has become more and more significant, as the motivation to breach privacy increases. Therefore, special care needs to be taken in the handling and especially publishing of data sets containing personal data, such that individuals can not be uniquely identified. \\
\\
On first sight it seems sufficient to simply remove or replace attributes from the data, that are directly identifying the individual. However, this pseudonymised published data may be de-anonymised by re-identifying individuals through quasi-identifiers (QID) and combining the data with external data bases or background knowledge \cite{CpppOfDataStreams}. Let us illustrate such an attack by means of an example:

\begin{example}[Re-identification attacks] Suppose there is a table T with attributes (postcode, birthplace, birthdate, health) that is being publicly available. An attacker might possess background information in the form of a table B = (name, postcode, birthplace, birthdate) about the individuals. Then a simple join of T and B on the common attributes might reveal the individuals identities and the attacker gains access to the sensitive attribute "health". The common attributes between T and B are called quasi-identifier (QID) in this case.
\end{example} 
\\
\noindent There has been extensive research in the field of privacy preserving data publishing in order to protect static data sets from these kinds of attacks. Among many, two major methods/algorithms evolved namely \textit{k-anonymity} \cite{kanonymity2002} and \textit{l-diversity} \cite{ldiversity2006}. However, with the amount of data steadily growing, a new field with need for research emerged, that is the field of \textit{continuous privacy preserving data publishing}. The challenge in a continuous setting is that there is no single static data set with a fixed number of records, but a stream of records arriving at varying velocities and speed, which is possibly never ending. Hence, the traditional \textit{k-anonymity} and \textit{l-diversity} can't be applied in a straightforward way \cite{CpppOfDataStreams}. Next to information loss it is essential for publishing a data stream, to publish the records in a timely way after it arrives, if possible synchronous to the arrival. That is because often the data loses it's value if it takes to long to be published. Therefore, there is a need for an anonymisation technique with low latency and low information loss. \\
\\
Additionally, in the past few years a great variety of frameworks emerged that allow for distributed, high-performing, always available and accurate data streaming applications, such a framework is the by the Apache Foundation published open source project \textit{Apache Flink}.
In this study we leveraged the Flink functionality to the widest extent possible in order to implement a variation of the \textit{k-anonymity} and \textit{l-diversity} algorithms, which is running in parallel, applicable to data streams and promises a minimal information loss. \\
\\
This report is structured as follows: Firstly, we will provide a short overview over existing research in the field and further define the problem and the existing algorithms \textit{k-anonymity} and \textit{l-diversity}. Subsequently, the paper describes how we adapted the algorithms to the streaming setting and define in detail the new algorithm. This is followed by a brief description of the architecture of the solution with respect to \textit{Apache Flink} functionality. The third section contains the results of an evaluation study of the algorithm mainly with respect to latency. The report will conclude with a discussion on strengths and weaknesses of the approach and further possibilities for research.