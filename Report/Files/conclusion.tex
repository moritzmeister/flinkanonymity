\noindent In this research project we provided a short survey about previously conducted research in the area of privacy preserving publishing of data streams. Subsequently, we proposed a very practical algorithm, that aims at tackling the information loss with increasing $k$ in $k$-Anonymity in previously conducted research, while at the same time providing a way to implement the algorithm in a parallelized way. \\
A wide variety of Flink's functionality was examined in order to determine their use for the anonymization task. It turned out that the windowing functionality in combination with triggers is particularly useful in order to keep track of equivalence classes. The only issue that turned out to be quite cumbersome, is to access the information of the tuples after they have been added to a window, for example to check if there already is a tuple of an individual present in the window. A successful implementation of this would probably require the  use of more advanced Flink window functions than the rather general GlobalWindow. \\
\\
Our empirical evaluation verifies the potential of the approach of keying by $QID$. While this introduces a delay trade off, our approach is robust to outliers meaning a lot can be won in terms of reduced information loss. On first sight, the analysis with respect to the parallelism does not look very promising, however, this would need further investigation in order to quantify the overhead introduced by Apache Flink through the parallelism, in order to make a reliable conclusion about whether there is a scenario when the parallelism can be fully exploited.\\
\\
Hence, the paper leads to a few interesting directions for future work:\\
1. An imaginable scenario where the parallelism of the algorithm plays a larger role would be a case where the quasi-identifier has a very large number generalizations. \\
2. Another interesting aspect would be the quantification of the information loss for different $QID$ in direct comparison to the algorithms of \citeA{CpppOfDataStreams}. This would also require further evaluation of the suggested methods of \textit{k}-anonymizing stuck tuples after a certain delay constraint mentioned in the discussion. However this needs to be done on exactly the same data streams for comparability, which at this moment was out of the scope of the project. \\
3. Additionally, we did not scale the project to the largest extent possible with Apache Flink due to constraints by running the simulations on local machines. Further scalability tests on big Flink clusters with larger network buffer are still necessary, in order to completely evaluate the effects of parallelization.